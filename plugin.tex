\section{¿Cómo trabaja un plugin para AutoCAD?.}

La propia empresa AutoDesk proporciona los materiales necesarios para que sus usuarios desarrollen sus propios plugins y puedan personalizar y extender las funcionalidades de sus herramientas. Con un enfoque comercial muy interesante, si un usuario consigue sentirse comodo y personalizar una herramienta de forma sencilla, conseguiras retenerle de una forma simple y conseguir que continue trabajando con tu herramienta de forma sistemática en el tiempo.

En concreto, la propia empresa AutoDesk justifica su elección de la plataforma .NET para dotar de estas capacidades al usuario de sus herramientas:

\begin{itemize}

\item{.NET es una herramienta de desarrollo estandarizada en el mercado y soportada por una gran compañía como es Microsoft.}

\item{.NET proporciona una gran colección de librerias y métodos listos para ser usados dentro del entorno de ejecución de .NET}

\item{Rápida curva de aprendizaje, en poco tiempo estas completando tus primeros desarrollos.}

\item{Buen soporte de tipos: tipos básicos y tipos más complejos, como colecciones.}

\item{Buen entorno de desarrollo, con sus versiones gratis y profesionales.}

\end{itemize}

La tecnología .NET es usada en la mayoría de los productos de la compañía AutoDesk. La mayoría de los productos exponen sus funcionalidades a través de una API bien documentada en tecnología .NET.

El proceso de desarrollo es bastante sencillo. Tan solo hay que seguir los siguientes pasos:

\begin{itemize}

\item{Referenciar en el proyecto las dos librerias a través de las que el producto AutoCAD expone sus servicios al entorno .NET: AcDbMgd.dll y AcMgd.dll.}

\item{Desarrollar el código del plugin con el lenguaje escogido y las librerias estandar proporcionadas por el entorno .NET, la API proporcionada por AutoCAD y cualquier API de terceros que se pueda necesitar.}

\item{No es necesario extender ninguna clase específica para crear el plugin. Cualquier clase pública es válida y tan sólo debe anotarse el método que vaya a contener toda la lógica del mismo con la anotación <<CommandMethod>>. Esta anotación permite a AutoCAD indentificar los nuevos comandos implementados en el plugin y hacerlos accesibles al usuario final.}

\item{Compilar el proyecto y generar un ensamblado final con todos los componentes, archivo .DLL}

\item{Cargar el ensamblado desde el entorno .NET a través del comando NETLOAD.}

\item{Invocar el comando o comandos definidos en el plugin.}

\end{itemize}

\begin{figure}[H]
\begin{center}
\includegraphics[height=8cm]{imgs/plugin}
\caption{Esquema del proceso de desarrollo de un plugin para AutoCAD \cite{AutoCAD-NET-API}.}
\end{center}
\end{figure}
