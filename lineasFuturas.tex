\section{Líneas futuras de trabajo.}

Una vez completado el alcance definido para el proyecto de Sistemas Informáticos se abre la cuestión de cuales pueden ser las posibles líneas futuras de trabajo para este sistema. Llegados a este punto y con el aprendizaje obtenido durante el desarrollo del mismo, se hace necesario replantear el proyecto inicial de construir un sistema independiente que limpie y optimice los diseños de nuevas infraestructuras para ser utilizadas en los simuladores de tiempo avanzado. 

¿Es realmente necesario que el sistema sea independiente? Quizás sea más efectivo proporcionar una serie de plugins para la herramienta AutoCad que permitan cubrir todas las etapas del sistema:

\begin{itemize}

\item{Etapa 1. Identificar y obtener la información geométrica de los diferentes elementos contenidos en un fichero DWG.}
\item{Etapa 2. Aplicar técnicas de aprendizaje automático y reconocimiento de formas para identificar las diferentes estructuras complejas contenidas en el fichero y adaptarlas y formatearlas a los requisitos de los diferentes simuladores.}
\item{Etapa 3: Escritura de esas nuevas estructuras adaptadas en un fichero con un formato reconocido por los distintos simuladores.}

\end{itemize}

Las líneas futuras de trabajo para la etapa 1 es continuar reconociendo tipos de objetos y estructuras adicionales y realizar baterías de pruebas más extensas con materiales reales.

Para las etapas 2 y 3 se hace necesario desarrollar algún tipo de prototipo que muestre la viabilidad de este planteamiento para posteriormente, si se dan las condiciones favorables, desarrollar los plugins que permitan completar el funcionamiento global del sistema.

Además, en las pruebas se ha detectado que la usabilidad del proceso de selección de capas es mejorable. Por ello, se plantea una línea de trabajo futura para esta funcionalidad. Esta línea de trabajo se abordaría de dos formas diferentes:

\begin{itemize}

\item{Selección a través de un fichero de configuración que contiene nombres de capas o patrones de nombres de capas que seleccionarán y serán incluidas por defecto en el proceso.}

\item{Selección a través de un cuadro de diálogo que muestra las capas preseleccionadas y las capas que
se pueden seleccionar permtiendo mover las capas entre ambas listas.}

\end{itemize}