\section{Contexto.}

El contexto de este proyecto de SSII se engloba dentro del amplio mundo del \textit{Aprendizaje Automatico} y del\textit{Reconocimiento de formas}. En concreto, en la aplicaci�n pr�ctica del amplio conocimiento desarrollado en ambas �reas, a la optimizaci�n y desarrollo del sistema de transporte aereo.

El coste de desarrollar nuevas y mejores infraestructuras para el sistema de transporte aereo es muy elevado. Requiere de grandes inversiones y los plazos de ejecuci�n suelen ser muy largos. Esto requiere que todas las inversiones sean analizadas y estudiadas en un alto grado de detalle antes de ser aprobadas y puestas en ejecuci�n para tratar de garantizar el �xito de las mismas.

Para ello, las organizaciones punteras y grandes compa��as del sector, estan desarrollando y trabajando con diferentes herramientas de simulaci�n en tiempo acelerado, que permiten simular y estudiar el comportamiento y el rendimiento de las nuevas infraestructuras sin necesidad de haberlas construido, de forma que se puede estimar el rendimiento que se obtendr� de las mismas, detectar posibles fallos o encontrar puntos de mejora previos a la ejecuci�n de las mismas.

Todas estas herramientas de simulaci�n toman como punto de partida ficheros de CAD \cite{CAD} con la descripci�n f�sica de la infraestructura a probar. Estos ficheros de CAD deben cumplir una serie de especificaciones muy concretas para poder ser interpretados correctamente por los simuladores y poder realizar las valoraciones adecuadas de cada uno de los dise�os. En la realizaci�n de estos dise�os CAD existe un peso importante del factor humano. Los dise�adores conocen las guias y tratan de aplicarlas pero en dise�os muy complejos, no siempre siguen la gu�a de estilo al completo y se ajustan al cien por cien a las especificaciones. 

Este hecho dificulta enormemente el procedimiento de trabajo con los simuladores puesto que exige revisar completamente y ajustar el fichero original ideado por el dise�ador a las especificaciones del mismo. Muchas veces esta tarea supone una gran cantidad de horas de trabajo del dise�ador y del personal encargado de la simulaci�n con el coste asociado que esto conlleva \cite{Mariscal-2005}.

Este proyecto de SSII forma parte de una soluci�n global que trata de minimizar ese trabajo, introduciendo procedimientos software autom�ticos que puedan realizar ese trabajo de limpieza y ordenaci�n de los dise�os de nuevas infraestructuras con la m�nima interacci�n humana posible.

Esta soluci�n global ya ha sido planteada y probada en \cite{Miguel-Munoz} y en alguna otra pr�ctica de Sistemas Inform�ticos \cite{Miguel-Munoz-SSII} \cite{Javier-Cabrera-SSII}

Dentro del mundo del CAD existe una empresa lider el mercado cuyos productos tienen una importante cuota de mercado y son considerados un estandar de facto, la empresa Autodesk con su software de dise�o asistido por ordenador AutoCAD como producto estrella, con sus m�ltiples variantes y adaptaciones a diferentes sectores del dise�o industrial y de la construcci�n. Todos los productos de esta compa��a trabajan con un formato de archivo denominado DWG \cite{DWG-file-history}. Es un formato propietario, con m�ltiples versiones, asociadas a las diferentes versiones de los productos que desde 1982 ha sacado Autodesk al mercado. 