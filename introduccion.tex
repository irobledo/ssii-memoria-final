\section{Contexto.}

El contexto de este proyecto de SSII se engloba dentro del amplio mundo del \textit{Aprendizaje Automático} y del \textit{Reconocimiento de formas}. En concreto, en la aplicación práctica del amplio conocimiento desarrollado en ambas áreas, a la optimización y desarrollo del sistema de transporte aéreo.

El coste de desarrollar nuevas y mejores infraestructuras para el sistema de transporte aéreo es muy elevado. Requiere de grandes inversiones y los plazos de ejecución suelen ser muy largos. Esto requiere que todas las inversiones sean analizadas y estudiadas en un alto grado de detalle antes de ser aprobadas y ejecutadas para tratar de garantizar el éxito de las mismas.

Para ello, las organizaciones punteras y grandes compañías del sector, están desarrollando y trabajando con diferentes herramientas de simulación en tiempo acelerado, que permiten simular y estudiar el comportamiento y el rendimiento de las nuevas infraestructuras sin necesidad de haberlas construido, de forma que se puede estimar el rendimiento que se obtendrá de las mismas, detectar posibles fallos o encontrar puntos de mejora previos a la ejecución de las mismas.

Todas estas herramientas de simulación toman como punto de partida ficheros de CAD \cite{CAD} con la descripción física de la infraestructura a probar. Estos ficheros de CAD deben cumplir una serie de especificaciones muy concretas para poder ser interpretados correctamente por los simuladores y poder realizar las valoraciones adecuadas de cada uno de los diseños. En la realización de estos diseños CAD existe un peso importante del factor humano. Los diseñadores conocen las guías y tratan de aplicarlas pero en diseños muy complejos, no siempre siguen la guía de estilo al completo y se ajustan al cien por cien a las especificaciones. 

Este hecho dificulta enormemente el proceso de trabajo con los simuladores puesto que exige revisar completamente y ajustar el fichero original ideado por el diseñador a las especificaciones del mismo. Muchas veces esta tarea supone una gran cantidad de horas de trabajo del diseñador y del personal encargado de la simulación con el coste asociado que esto conlleva \cite{Mariscal-2005}.

Este proyecto de SSII forma parte de una solución global que trata de minimizar ese trabajo, introduciendo procedimientos software automáticos que puedan realizar ese trabajo de limpieza y ordenación de los diseños de nuevas infraestructuras con la mínima interacción humana posible.

Esta solución global ya ha sido planteada y probada en \cite{Miguel-Munoz} y en alguna otra práctica de Sistemas Informáticos \cite{Miguel-Munoz-SSII} \cite{Javier-Cabrera-SSII}

Dentro del mundo del CAD existe una empresa líder cuyos productos tienen una muy importante cuota de mercado y son considerados un estándar de facto. Son la empresa Autodesk y su software de diseño asistido por ordenador AutoCAD , con sus múltiples variantes y adaptaciones a diferentes sectores del diseño industrial y de la construcción. Todos los productos de esta compañía trabajan con un formato de archivo denominado DWG \cite{DWG-file-history}. Es un formato propietario, con múltiples versiones, asociadas a las diferentes versiones de los productos que desde 1982 ha sacado Autodesk al mercado. Tal y como se ha mencionado, el formato del fichero DWG es cerrado y dado que se convirtió en un estándar de facto, Autodesk publicó una variante simplificada del mismo, para poder interoperar con otros software de CAD diferentes. Este formato, cuya especificación es abierta, es el formato de fichero DXF \cite{DXF-file-history}. 

La solución global ya planteada \cite{Miguel-Munoz} se diseño y construyo para trabajar sobre ficheros con el formato DXF. Con el paso del tiempo, el formato del fichero DXF no ha evolucionado tan rápido como el formato de fichero DWG, y en la transformación no existe una traslación directa de toda la información. Es por ello que se plantea este proyecto de SSII para poder abordar una solución conceptualmente igual a la ya planteada pero técnicamente distinta, utilizando como fuente de información de partida, el propio formato del fichero DWG.

En concreto, el objetivo de este proyecto es obtener un sistema informático capaz de extraer toda la información necesaria de un fichero con un formato DWG, ordenarla y guardarla en un formato accesible, como puede ser un formato XML, de forma que otros sistemas informáticos sean capaces de tratar esa misma información y adecuarla al formato requerido por los diferentes simuladores de tiempo avanzado. 

Durante el desarrollo de este proyecto de Sistemas Informáticos se ha procedido a:

\begin{itemize}

\item{Estudiar la estructura del formato de fichero DWG.}
\item{Buscar y seleccionar las herramientas más apropiadas para manipular el formato de fichero DWG y poder obtener toda la información posible de los distintos elementos gráficos y no gráficos contenidos en el mismo.}
\item{Desarrollar un proceso de extracción de toda la información básica de los diferentes elementos y serialización de la misma a un fichero de intercambio que pueda ser utilizado por otros procesos para abordar las etapas de reconocimiento de formas y de aprendizaje automático.}
\item{Realizar pruebas de extracción sobre diferentes ejemplos de ficheros con formato DWG.}

\end{itemize}

A lo largo de este documento se realiza una descripción de cada una de las etapas anteriores, de la solución final alcanzada y se esbozan posibles futuras líneas de trabajo posteriores a la realización de este proyecto. En los anexos se puede encontrar información sobre como instalar y utilizar la herramienta desarrollada y un ejemplo con los resultados de una ejecución de la misma.
