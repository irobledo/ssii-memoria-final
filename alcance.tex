\section{Definición del alcance. Requisitos funcionales.}

Dada la complejidad de los ficheros DWG, que contienen gran cantidad de información diferente, se ha definido el siguiente alcance para el proyecto de Sistemas Informáticos:


\begin{itemize}

\item{Req 1: Abrir y manipular un fichero en formato DWG en modo lectura.}
\item{Req 2: Identificar las capas existentes dentro del fichero.}
\item{Req 3: Identificar los objetos de tipo punto existentes dentro del fichero.}
\item{Req 4: Identificar los objetos de tipo línea existentes dentro del fichero.}
\item{Req 5: Identificar los objetos de tipo polilínea existentes dentro del fichero.}
\item{Req 6: Identificar los objetos de tipo arco existentes dentro del fichero.}
\item{Req 7: Identificar los siguientes atributos para cada uno de los elementos anteriores: color, grosor de la línea, capa a la que pertenecen}.
\item{Req 8: Almacenar la información obtenida de cada uno de los elementos en una estructura de objetos en memoria que en futuros desarrollos pueda ser utilizada}.
\item{Req 9: Exportar a un fichero de intercambio el contenido de esa estructura de objetos en memoria para ser utilizado por otros procesos, y más en concreto, por el proceso visualizador que se esta desarrollando en otro proyecto de SSII.}
\item{Req 9: Permitir al usuario indicar la ruta donde quiere guardar el fichero de intercambio}.
\item{Req 10: Permtir al usuario seleccionar que capas deben ser procesadas en busca de los objetos y cuales deben ser omitidas del proceso de búsqueda.}.

\end{itemize}
