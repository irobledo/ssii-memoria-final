\section{Desarrollo.}
El Sistema Informático desarrollado se compone de 3 módulos / elementos que de forma conjunta proveen de la funcionalidad necesaria a AutoCAD para exportar los elementos del fichero DWG a un fichero en un formato XML. Estos módulos se apoyan en los siguientes componentes que el propio AutoCAD expone a disposición de los desarrolladores:

\begin{itemize}

\item{AutoDesk.AutoCAD.Runtime: este módulo inicializa los servicios de la aplicación cliente y las factorías que producen los diferentes objetos para ser utilizados dentro del entorno .NET.}

\item{AutoDesk.AutoCAD.ApplicationServices: este módulo permite la interacción con la aplicación AutoCAD. Provee acceso a los módulos adicionales como el editor, el gestor de documentos, ventana principal, etc.}

\item{AutoDesk.AutoCAD.DatabaseServices: esto módulo permite la interacción con el fichero de dibujo AutoCAD (DWG). Permite el acceso al header del mismo, a las tablas de símbolos, a las tablas de registros de entidades y a los diferentes objetos que componen el dibujo.}

\item{AutoDesk.AutoCAD.Geometry: esto modulo permite la interacción con el servicio de geometría de AutoCAD que se encarga de la gestión de las referencias espaciales.}

\item{AutoDesk.AutoCAD.EditorInput: esto modulo permite la interacción con el editor a través del cual se realiza la interacción con el usuario en ambos sentidos.}

\end{itemize}

Dado que se desarrolla una aplicación que es ejecutada dentro de otra aplicación se ha optado por un diseño muy sencillo, sin la aplicación de los clásicos patrones de diseño orientados a objetos que aportan capas y métodos para desacoplar los diferentes objetos, favoreciendo la reusabilidad y el mantenimiento. En este caso, la solución esta muy acoplada, con todos los objetos muy integrados haciéndola un poco más eficiente en tiempo de ejecución. Aunque esto puede hacer más complicado el mantenimiento y actualización de la misma, en realidad, este factor se ve minimizado dado que sólo 3 módulos / elementos los que se han desarrollado:

\begin{itemize}

\item{dwgElementos: este módulo contiene las clases necesarias para representar y almacenar en memoria la información contenida en un fichero con formato DWG. En concreto esta formado por las clases:}

\begin{itemize}

\item{dwgPunto: representa un punto de un plano de AutoCAD. Sus miembros principales son las 3 coordenadas espaciales: X,Y,Z y el color con el que debe ser pintado.}
\item{dwgLinea: representa una línea recta de un plano de AutoCAD. Sus miembros principales son el identificador del punto origen de la línea, el identificador del punto final de la línea, el ancho y el color con el que debe ser pintada.}
\item{dwgArco: representa un arco de un plano de AutoCAD. Sus miembros principales son el identificador del punto central del arco, el ángulo inicial del mismo, el ángulo final y el radio. También contiene la asociación con el conjunto de líneas rectas equivalentes.}
\item{dwgPolylinea: representa una polilínea de un plano de AutoCAD. Sus miembros principales son los que  contienen la asociación con el conjunto de líneas rectas y arcos equivalente.}
\item{dwgCapa: representa una capa de un plano de AutoCAD.}
\item{dwgFile: representa un fichero DWG conteniendo toda la información del mismo relativa a capas, puntos, líneas, polilíneas y arcos.}

\end{itemize}

\item{dwgTools: este módulo contiene las clases de apoyo y herramientas necesarias para completar el proceso de extracción de la información del fichero DWG. En concreto, esta formado por las clases:}

\begin{itemize}

\item{herramientasCurvas: contiene las clases y métodos necesarios para transformar una curva en un conjunto de líneas rectas equivalentes.}
\item{exportXML: contiene las clases y métodos necesarios para exportar el contenido representado en memoria a través del módulo dwgElementos a un fichero de texto.}

\end{itemize}

\item{dwgDecoder: módulo principal que implementa el plugin de AutoCAD apoyándose en los dos módulos anteriores.
El proceso de decodificación se compone de 4 etapas. La primera de ellas corresponde con la configuración del proceso donde se solicita al usuario si desea un log del proceso y el grado de detalle de la información del mismo, la ruta donde se almacenará el fichero de salida y si desea seleccionar manualmente las capas a incluir en el proceso de extracción. A continuación se muestra el diagrama de flujo correspondiente a esta etapa:

\begin{figure}[H]
\begin{center}
\includegraphics[width=0.65\textwidth]{imgs/diagramaFlujo}
\caption{Diagrama de flujo de la etapa de configuración.}
\end{center}
\end{figure}

En una segunda etapa se examinan todas las capas existentes en el fichero y si el usuario ha indicado que quiere seleccionar manualmente las capas a procesar se le va preguntando por cada una ellas para que decida si su información debe ser interpretada. Si el usuario no ha indicado que quiere hacer una selección manual se obtiene e interpreta la información de todas las capas. 

Toda la interacción con el usuario se realiza a través del módulo EditorInput que expone AutoCAD. Se informa al usario a través de un mensaje en la consola de la capa que va a ser procesada y se solicita que indique si desea incluirla o no en el proceso. Toda la recepción y validación de la información a través de la consola se realiza también con los servicios facilitados por el módulo EditorInput. A continuación se muestra el diagrama de flujo correspondiente a esta etapa:

\begin{figure}[H]
\begin{center}
\includegraphics[width=0.65\textwidth]{imgs/diagramaFlujo2}
\caption{Diagrama de flujo de la etapa de lectura de capas.}
\end{center}
\end{figure}

En una tercera etapa se examinan todos objetos existentes dentro del fichero DWG. Si el objeto esta en una de las capas que debe ser procesada, se interpreta la información del objeto y se incorpora al contenido del objeto de tipo dwgFile. En el caso de los objetos polilínea y arco se descomponen en líneas simples antes de ser incorporados al objeto de tipo dwgFile, manteniendo siempre la relación con el objeto inicial para que puedan ser analizadas y tratadas en conjunto. A continuación se muestra el diagrama de flujo correspondiente a esta etapa:

\begin{figure}[H]
\begin{center}
\includegraphics[width=0.65\textwidth]{imgs/diagramaFlujo3}
\caption{Diagrama de flujo de la etapa de lectura de objetos.}
\end{center}
\end{figure}

En una cuarta y última etapa se vuelca todo el contenido del objeto dwgFile a un fichero de texto en formato XML almacenándolo en la ruta indicada por el usuario durante el proceso de configuración del proceso de extracción. El formato de este fichero XML esta detallado en el apéndice 2 de este documento mediante el correspondiente XML Schema.
}

\end{itemize}

A continuación se muestran los diagramas UML del módulo dwgElementos y del sistema completo con todos los elementos relacionados:

\begin{sidewaysfigure}
\begin{center}
\includegraphics[width=0.9\textwidth]{imgs/dwgElementos}
\caption{Diagrama UML del módulo dwgElementos.}
\end{center}
\end{sidewaysfigure}

\begin{sidewaysfigure}
\begin{center}
\includegraphics[width=0.9\textwidth]{imgs/dwgDecoder}
\caption{Diagrama UML del sistema dwgDecoder.}
\end{center}
\end{sidewaysfigure}