\section{Pruebas.}

Para verificar el correcto funcionamiento del plugin se realizan diferentes pruebas de exportación de diferentes grados de dificultad. Las primeras pruebas sencillas consisten en la exportación del contenido de ficheros DWG con contenidos básicos:

\begin{itemize}

\item{Fichero con una o varias líneas.}
\item{Fichero con una polilínea}
\item{Fichero con varias polilíneas}
\item{Fichero con un arco}
\item{Fichero con varios arcos}
\item{Fichero combinando todos los elementos anteriores en una sola capa.}
\item{Fichero combinando todos los elementos anteriores en varias capas.}

\end{itemize}

Los resultados de estas pruebas son satisfactorios.

En una segunda fase se obtiene a través del distribuidor de AutoCAD varios ejemplos de ficheros de complejidad media con combinaciones de capas y objetos más complejos.

Los resultados de estas pruebas también son satisfactorios. 

Superados los dos primeros conjuntos de pruebas, se procede a probar con uno de los ficheros originales, utilizados en \cite{Miguel-Munoz}, que se ha establecido como la prueba de validación y verificación del correcto funcionamiento del proceso de exportación. 

Los resultados son totalmente satisfactorios.

En una última fase de pruebas se quiere verificar la consistencia del proceso ante ficheros muy complejos. Se obtiene a través del distribuidor de AutoCAD dos ejemplos de ficheros del diseño de una planta de un edificio nuevo con miles de objetos de distintos tipos en su interior y gran cantidad de capas. 

En esta última fase los resultados son también satisfactorios completándose correctamente la fase de exportación en un tiempo bastante reducido.

Por tanto, tras todas estas pruebas, puede considerarse verificado el correcto funcionamiento del plugin y la consistencia del mismo.